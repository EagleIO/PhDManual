% Options for packages loaded elsewhere
\PassOptionsToPackage{unicode}{hyperref}
\PassOptionsToPackage{hyphens}{url}
%
\documentclass[
]{book}
\usepackage{amsmath,amssymb}
\usepackage{lmodern}
\usepackage{iftex}
\ifPDFTeX
  \usepackage[T1]{fontenc}
  \usepackage[utf8]{inputenc}
  \usepackage{textcomp} % provide euro and other symbols
\else % if luatex or xetex
  \usepackage{unicode-math}
  \defaultfontfeatures{Scale=MatchLowercase}
  \defaultfontfeatures[\rmfamily]{Ligatures=TeX,Scale=1}
\fi
% Use upquote if available, for straight quotes in verbatim environments
\IfFileExists{upquote.sty}{\usepackage{upquote}}{}
\IfFileExists{microtype.sty}{% use microtype if available
  \usepackage[]{microtype}
  \UseMicrotypeSet[protrusion]{basicmath} % disable protrusion for tt fonts
}{}
\makeatletter
\@ifundefined{KOMAClassName}{% if non-KOMA class
  \IfFileExists{parskip.sty}{%
    \usepackage{parskip}
  }{% else
    \setlength{\parindent}{0pt}
    \setlength{\parskip}{6pt plus 2pt minus 1pt}}
}{% if KOMA class
  \KOMAoptions{parskip=half}}
\makeatother
\usepackage{xcolor}
\IfFileExists{xurl.sty}{\usepackage{xurl}}{} % add URL line breaks if available
\IfFileExists{bookmark.sty}{\usepackage{bookmark}}{\usepackage{hyperref}}
\hypersetup{
  pdftitle={MSU PhD Roles and Policy Manual},
  pdfauthor={Industrial and Organizational Program Faculty},
  hidelinks,
  pdfcreator={LaTeX via pandoc}}
\urlstyle{same} % disable monospaced font for URLs
\usepackage{longtable,booktabs,array}
\usepackage{calc} % for calculating minipage widths
% Correct order of tables after \paragraph or \subparagraph
\usepackage{etoolbox}
\makeatletter
\patchcmd\longtable{\par}{\if@noskipsec\mbox{}\fi\par}{}{}
\makeatother
% Allow footnotes in longtable head/foot
\IfFileExists{footnotehyper.sty}{\usepackage{footnotehyper}}{\usepackage{footnote}}
\makesavenoteenv{longtable}
\usepackage{graphicx}
\makeatletter
\def\maxwidth{\ifdim\Gin@nat@width>\linewidth\linewidth\else\Gin@nat@width\fi}
\def\maxheight{\ifdim\Gin@nat@height>\textheight\textheight\else\Gin@nat@height\fi}
\makeatother
% Scale images if necessary, so that they will not overflow the page
% margins by default, and it is still possible to overwrite the defaults
% using explicit options in \includegraphics[width, height, ...]{}
\setkeys{Gin}{width=\maxwidth,height=\maxheight,keepaspectratio}
% Set default figure placement to htbp
\makeatletter
\def\fps@figure{htbp}
\makeatother
\setlength{\emergencystretch}{3em} % prevent overfull lines
\providecommand{\tightlist}{%
  \setlength{\itemsep}{0pt}\setlength{\parskip}{0pt}}
\setcounter{secnumdepth}{5}
\usepackage{booktabs}
\ifLuaTeX
  \usepackage{selnolig}  % disable illegal ligatures
\fi
\usepackage[]{natbib}
\bibliographystyle{apalike}

\title{MSU PhD Roles and Policy Manual}
\author{Industrial and Organizational Program Faculty}
\date{2022-05-23}

\begin{document}
\maketitle

{
\setcounter{tocdepth}{1}
\tableofcontents
}
\hypertarget{io-psychology-doctoral-program-roles-and-policies}{%
\chapter{I/O Psychology Doctoral Program Roles and Policies}\label{io-psychology-doctoral-program-roles-and-policies}}

\includegraphics{manual.png}

As the I/O Doctoral Program goes live, we are in need of a handbook detailing all the key procedures and policies for the I/O doctoral and MA program experience. The goal is to have a final draft of the handbook by December 1st 2020, or the deadline for our first official doctoral application cycle. We need to have formal procedures for the following areas: (1) Doctoral Student Selection, Chapter \ref{select}, (2) I/O Program (MA and PhD) Probation Standards, Chapter \ref{develop}, (3) Prospective Comprehensive Exams, Chapter \ref{comps}, (4) Student Mentorship and Program Culture, Chapter \ref{culture}, and (5) Dissertation Process, Chapter \ref{dissertation}. Here are some general handbooks from other programs:

\begin{itemize}
\tightlist
\item
  \href{https://www.clemson.edu/cbshs/departments/psychology/files/pdf/handbook-psychology-graduate-current.pdf}{Clemson}
\item
  \href{https://graduate.ucf.edu/wp-content/uploads/2019/05/Industrial-and-Organizational-Psychology-PhD.pdf}{UCF}
\item
  \href{https://artsandsciences.utulsa.edu/wp-content/uploads/sites/6/2015/09/2017-I-O-Handbook.pdf}{Tulsa}
\end{itemize}

It would be great help if each of us could volunteer to develop one or more of these policies and then educate the other faculty on the process. We can submit each for review, feedback, and then finalize for our second official incoming class. We could each carve out the procedure \textbf{over the summer}, review as a group \textbf{at our first I/O meeting} in the Fall, then \textbf{update and finalize} by December 1st. Below is a brief outline of each policy with some examples from other institutions.

\hypertarget{develop}{%
\chapter{I/O Program (MA and PhD) Expectations, Probation and Development Standards}\label{develop}}

We need to find a better way to set high expectations early in the program, reinforce these via class, and quickly exit poor performers showing little investment in personal improvement. This should start with a set of agreed-upon expectations and extracurriculars which we convey during our welcoming event. I know stats and WAM are heavy hitters in the first semester, and both Kev and Val do a good job setting the tone. By crystallizing this in the handbook it would be a way to officially promote certain standards as a program. For instance, USF's handbook has a listing of expectations for doc students which includes reading TIP, presenting research at SIOP, attending conferences, and getting involved in supporting the program. Perhaps we could come up with a general set of expectations for MA students as well, such as hours of reading per week, self-driven learning, active discussion, and participation in local events? Just a thought.

For retention standards, we have had several students over the past years (e.g., Aristotle, Alan, etc\ldots) manage to skate through by maintaining grades at the 3.0 level. Need to update terms for dismissal so we can more quickly identify poor performers early on, get the warning messages out, and basically signal earlier they are at risk for termination. This will make it easier to remove from the program after the first year before they have expended too many resources into a degree that will not serve them well.

Finally, we need an annual evaluation process for doctoral students to track accomplishments in research, professional development, and program engagement. Think of it as capturing performance beyond the classroom for developmental and evaluative purposes. Many programs hold such regular reviews between advisors and doctoral students each year.

Combining the above, I think we need (a) general expectations for grad students in the program handbook which speak to the behaviors we like to see in MA and PhD students, (b) set probationary status for GPA \textless{} 3.25, (c) a template and development plan for those falling below 3.25 or earning straights Bs early on, and (D) an anonymous vote for retention or not if the student is still below a 3.25 two semesters in a row. On top of this, perhaps a template for how to encourage poor students to drop out (e.g., how to say this is a bad fit) and then resources we can provide them for finding another program (e.g., HR or MBA program perhaps). It is ideal if these procedures are concrete yet holistic to allow compensatory judgments.

For annual doctoral review, we need a set of dates and a form for us to fill out individually with each of our doctoral students. Several programs provide such forms in their handbooks.

\begin{itemize}
\tightlist
\item
  \href{http://catalog.utc.edu/content.php?catoid=15\&navoid=465\#Academic_Dismissal}{Chattanooga}
\item
  \href{https://www.fit.edu/media/site-specific/wwwfitedu/college-of-psychology-and-liberal-arts/io/documents/2019-IO-Student-Handbook.pdf}{Florida Tech Manual with Doc Evaluation Form}
\end{itemize}

\hypertarget{comps}{%
\chapter{Comprehensive Exams}\label{comps}}

We will be experimenting with a prospective exam model which is heavily built around questions which target their preparation for dissertation work. This means questions targeted at how well they have integrated areas related to their dissertation, prepared methods/analytic approaches to target their question, critically evaluated key areas and known limitations, and show understanding of how their topic will push forward scholarly work.Note this exam is an exit point exam meaning it should be a real possibility for some students to fail and depart with their MA degree. Rogelberg indicated this happened at UNC meaning it still is intended to be a rigorous process. For comps we need a procedure, set of criteria, timeline, and decision on how many phases to complete. I have found many prospective comp policies from other doctoral programs which are very thorough and provide good examples, such as:

\begin{itemize}
\tightlist
\item
  \href{https://www.mcgill.ca/socialwork/prospective/phd/comprehensive-exam}{McGill}
\item
  \href{https://www.umass.edu/communication/node/861}{UMass}
\item
  \href{http://gradschool.psu.edu/graduate-education-policies/gcac/gcac-600/gcac-604-qualifying-exam/}{PennState}
\end{itemize}

An example of a traditional I/O comp exams are as follows:

\begin{itemize}
\tightlist
\item
  \href{https://liberalarts.tamu.edu/wp-content/uploads/sites/2/2018/02/I-O-area-comps-policy-Nov-16th-2015.pdf}{Texas A\&M}
\end{itemize}

\hypertarget{culture}{%
\chapter{Student I/O Group, Mentorship, and Program Culture}\label{culture}}

Maybe shift to ``how to navigate the program'' - initial PhD advisor (upon matriculation) versus selecting one in pursuit of the PhD

Policies and procedures for an official I/O student group (perhaps Eagle I/O is sufficient) with officers and roles charged with social events, mentoring, speakers, assembling student resources, putting together SIOP reception, etc\ldots{} Basically a charter and set of annual tasks for a select set of I/O Doctoral and MA student officers to complete to improve the strength of the bonds between the students and extracurricular opportunities to strengthen their vocational self-concepts. Further, we need to make sure one faculty member (John for now) is always overseeing the mentorship and eagle I/O group to ensure its continuity. Here are several examples of such groups and their offerings at other I/O programs

Characteristics we value (self-initiative, internal locus-of-control), engagement with alumni and bringing external speakers on campus, encouragement of METRO \& SIOP, but we're not going to delineate a process to build and maintain culture

\begin{itemize}
\tightlist
\item
  \href{http://ohdcc.org/}{Columbia's OHDCC group}
\item
  \href{https://io.gmu.edu/iopsa}{George Mason I/O Student Group}
\item
  \href{https://artsandsciences.utulsa.edu/wp-content/uploads/sites/6/2015/02/GRASP-Handbook-2015-2016.pdf}{CSUSB's Student IO Group}
\item
  \href{https://artsandsciences.utulsa.edu/wp-content/uploads/sites/6/2015/02/GRASP-Handbook-2015-2016.pdf}{Student Advice Guide from GRASP at Tulsa}
\item
  \href{http://iopac.colostate.edu/}{Colorado's I/O Group}
\end{itemize}

We already have an official mentorship guide and procedure. However, the program is still growing organically through John's work and may expand to include other student responsibilities. Given this, perhaps a high-level, one page overview of current procedures, expectations, and roles on the mentorship program for the handbook? Something like a general overview to guide PhD and MA students on scheduling, involvement, and primary tasks.

\hypertarget{dissertation}{%
\chapter{Dissertation Process}\label{dissertation}}

The graduate school has a general dissertation process, so we could follow their guidelines. However, it would be helpful to spell out our own logistics for dates, committee assemblage, proposal process, and links within the I/O document so students can reference this if they have questions. Further, I put the general dissertation process outline in the PAD which can be reviewed and expanded upon with more specific procedural details. Below are some examples of programs which have their dissertation process spelt out.

\begin{itemize}
\tightlist
\item
  \href{https://www.utc.edu/doctorate-learning-leadership/doctoralguide/dissertationprocess/finaldefense.php}{Chattanooga}
\item
  \href{https://www.lsu.edu/hss/psychology/grad/io_graduate_handbook_2019-11-07.pdf}{Louisiana State}
\end{itemize}

  \bibliography{book.bib,packages.bib}

\end{document}
